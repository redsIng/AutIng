\documentclass{article}
\usepackage[italian]{babel}
\usepackage{amssymb,theorem}

%%%%%%%%%% Start TeXmacs macros
\newcounter{nnanswer}
\def\thennanswer{\unskip}
{\theorembodyfont{\rmfamily}\newtheorem{answer*}[nnanswer]{Answer}}
\newcounter{nnquestion}
\def\thennquestion{\unskip}
{\theorembodyfont{\rmfamily}\newtheorem{question*}[nnquestion]{Question}}
%%%%%%%%%% End TeXmacs macros

\begin{document}

{\center{\part*{DOMANDE TEORIA OSC I}}}
\begin{question*}
  Descrivere brevemente l'equazione di Hamilton-Jacaobi Bellman. Dimostrare
  che HJB fornisce condizioni necessarie di ottimalit{\`a}.
\end{question*}

\begin{answer*}
  L'equazione di Hamilton Jacobi Bellman {\`e}:
  \[ \left\{\begin{array}{l}
       - \frac{\partial V}{\partial t}^{\ast} (x (t), t) = \min_{u (t)}
       \left\{ l (x (t), u (t), t) + \frac{\partial V}{\partial x}^{\ast} (x
       (t), t) f (x (t), u (t), t) \right\}\\
       V^{\ast} (x, T) = m (x)
     \end{array}\right. \quad \forall x \in \mathbb{R}^n, t \in [t_0, T] \]
  Permette di rappresentare la funzione valore ottima di un problema di
  controllo ottimo di Bolza:
  \[ \left\{\begin{array}{l}
       \min_u J (u) = \min_u \left\{ \int_{t_0}^T l (x (\tau), u (\tau), \tau)
       d \tau + m (x (T)) \right\}\\
       \dot{x} = f (x, u, t)\\
       x (t_0) = x_0
     \end{array}\right. \]
  Generalmente questa equazione non {\`e} lineare e la sua soluzione {\`e}
  proprio la funzione valore ottima che rispetta l'equazioene di Hellman del
  problema dato e, quindi, rappresentante le decisioni ottime da effettuare
  partendo da un determinato stato e tempo iniziale. In particolare,
  l'equazione HJB fornisce condizioni necessarie di ottimalit{\`a}. A tale
  scopo definiamo la funzione valore per il problema di Bolza:
  \[ V^{\ast} (x (t), t) = \min_{u (\tau), \tau \in [\tau .T]} \left\{
     \int_t^T l (x (s), u (s), s) d s + m (x (T)) \right\} \]
  \[ = \min_{u (\tau), \tau \in [t, T]} \left\{ \int_t^{t + \Delta t} l (x
     (s), u (s), s) d s + \int_{t + \Delta t}^T l (x (s), u (s), s) d s + m (x
     (T)) \right\} \]
  \[ = \min_{u (\tau), \tau \in [t, T]} \left\{ \int_t^{t + \Delta t} l (x
     (s), u (s), s) d s + V^{\ast} (x (t + \Delta t), t + \Delta t) \right\}
  \]
  Da questa espressione ricaviamo la sua corrispondente espressione
  differenziale sviluppando al primo ordine la funzione valore:
  \[ \left\{\begin{array}{l}
       0 = \min_{u (t)} \left\{ l (x (t), u (t), t) + - \frac{\partial
       V}{\partial t}^{\ast} (x (t), t) + \frac{\partial V}{\partial x}^{\ast}
       (x (t), t) f (x (t), u (t), t) \right\}\\
       V^{\ast} (x, T) = m (x)
     \end{array}\right. \quad \forall x \in \mathbb{R}^n, t \in [t_0, T] \]
  Dato che dobbiamo determinare la soluzione per ogni istante di tempo e per
  ogni stato, si ottiene l'equazione HJB:
  \[ \left\{\begin{array}{l}
       - \frac{\partial V}{\partial t}^{\ast} (x (t), t) = \min_{u (t)}
       \left\{ l (x (t), u (t), t) + \frac{\partial V}{\partial x}^{\ast} (x
       (t), t) f (x (t), u (t), t) \right\}\\
       V^{\ast} (x, T) = m (x)
     \end{array}\right. \quad \forall x \in \mathbb{R}^n, t \in [t_0, T] \]
\end{answer*}

\

\end{document}
