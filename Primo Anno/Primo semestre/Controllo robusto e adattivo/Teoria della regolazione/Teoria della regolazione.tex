\documentclass{article}
\usepackage[english]{babel}
\usepackage{graphicx}
\graphicspath{{Images/}}
\usepackage{geometry}
\geometry{a4paper, top=2cm, bottom=2cm, left=1.0cm, right=1.0cm}
\usepackage{amsthm}
\usepackage{amsmath}
\usepackage{fixltx2e}
\usepackage{amsfonts}
\usepackage{xcolor}
\usepackage{verbatim}
\usepackage{amssymb}
\newtheorem{theorem}{\textbf{Teorema}}
\newtheorem{corollary}{\textbf{Corollario}}
\theoremstyle{definition}
\newtheorem{definition}{\textbf{Definizione}}

\title{Toeria della regolazione}
\author{Lorenzo Rossi}
\begin{document}
\maketitle
Nella teoria della regolazione consideriamo un sistema lineare affetto da disturbi e tale che la sua uscita deve inseguire asintoticamente un segnale di riferimento noto.
\begin{equation*}
	\Sigma:\begin{cases}
		\dot{x}=Ax+Bu+Pd \\
		e=Cx+Qd
	\end{cases}
\end{equation*}
con \(x(t)\in\mathbb{R}^{n},u(t)\in\mathbb{R}^{m},e(t)\in\mathbb{R}^{p},d(t)\in\mathbb{R}^{r}\) e le matrici \(A\in\mathbb{R}^{n\times n},B\in\mathbb{R}^{n\times m},P\in\mathbb{R}^{n\times p},C\in\mathbb{R}^{p\times n},Q\in\mathbb{R}^{p\times r}\) note e costanti.\newline
In questo sistema si identifica:
\begin{itemize}
	\item \(d(t)\) che rappresenza un segnale esogeno composto da una componente del disturbo associato al processo e una componente dei segnali di riferimento. La sua dinamica è descritta da un sistema lineare:\begin{equation*}
		      \Sigma_{d}=S d\quad S\in\mathbb{R}^{r\times r},d(t)=\begin{bmatrix}
			      w(t) \\r(t)
		      \end{bmatrix}\in\mathbb{R}^{r}
	      \end{equation*}
	\item \(e(t)\) è l'errore di inseguimento del comportamento del sistema rispetto al comportamento ideale. Di norma vogliamo che si raggiunga l'obiettivo di \textbf{regolazione a zero}: l'errore deve convergere a zero tramite un controllo \(u(t)\) opportuno. Inoltre, la specifica di regolazione a zero implica che i disturbi non influenzano il comportamento del sistema e l'uscita \(y=Cx(t)\) insegue asintoticamente il segnale di riferimento \(r(t)=-Q d(t)\)
\end{itemize}
Il controllore \(u(t)\) necessario per la regolazione a zero può essere ottenuto in due modi:
\begin{itemize}
	\item \textbf{Controllore statico in feedback dallo stato \(x(t)\)}: Supponiamo che \(x(t)\) sia lo stato e \(d(t)\) sia il segnale esogeno, entrambi misurati. Allora si progetta la legge di controllo \(u=K x+L d\)
	      \begin{figure}[h]
		      \centering
		      \includegraphics[scale=0.3]{uStatico.drawio.png}
	      \end{figure}
	\item \textbf{Controllore dinamico dall'errore \(e(t)\)}: questo controllore non necessita che i segnali \(x(t),d(t)\) siano misurati, ma si costruisce un osservatore la cui uscita viene utilizzata per progettare un controllo \(u(t)\).\begin{equation*}
		      \begin{cases}
			      \dot{\chi}= F\chi+G e \\
			      u=H\chi
		      \end{cases}
		      \quad F\in\mathbb{R}^{\nu\times\nu},G\in\mathbb{R}^{\nu\times\nu}\text{note e costanti}
	      \end{equation*}
          \begin{figure}[h]
            \centering
            \includegraphics[scale=0.3]{uDynamic.drawio.png}
        \end{figure}
\end{itemize}
Nella teoria di regolazione ci si riferisce principalemnte a due tipi di problemi.
\begin{definition}{\textbf{Differenze}}\\
    La differenza con un problema di controllo \(H_{\infty }\) è che nel problema \(H_{\infty }\) si vuole attenuare l'effetto del disturbo \(d\) sul segnale di prestazioni \(z\) in termini di guadagno \(L_{2}\) progettando un controllore che garantisca stabilità asintotica del sistema a ciclo e un guadagno \(L_{2}\) intresso-uscita minore di \(\gamma \); nel prolma di regolazione si impone una struttura sulla forma del segnale \(d\) e si richiede che si abbia un errore asintoticamente nullo.
\end{definition}
\begin{definition}{\textbf{Problema di regolazione a full information}}\\
	Considerato il sistema:\(\begin{cases}
		\dot{x}=Ax+Bu+Pd \\
		e=Cx+Qd
	\end{cases}\) affetto da disturbi generati dall'esosistema \(\dot{d}=Sd\) interconnesso con il controllore \(u=K x+L d\). Il \textbf{problema di regolazione a informazione completa} è quello di determinare le matrici \(K,L\) dek controllore tali che siano soddisfatte:
    \begin{itemize}
		\item \textbf{\emph{Stabilità} (S)}:Il sistema \(\dot{x}=(A+BK)x\) sia asintoticamente stabile;
		\item \textbf{\emph{Regolazione} (R)}: tutte le traiettorie del sistema \(\begin{cases}
			      \dot{d}=Sd              \\
			      \dot{x}=(A+BK)x+(BL+P)d \\
			      e=Cx+Qd
		      \end{cases}\) siano tali che \(\lim_{t\rightarrow\infty}e(t)=0\)
	\end{itemize}
\end{definition}
\begin{definition}{\textbf{Problema di regolazione con retroazione dall'errore}}\\
    Considerato il sistema:\(\begin{cases}
		\dot{x}=Ax+Bu+Pd \\
		e=Cx+Qd
	\end{cases}\) affetto da disturbi generati dall'esosistema \(\dot{d}=Sd\) interconnesso con il controllore \(\begin{cases}
        \dot{\chi}=F\chi + Ge\\
        u=H\chi
    \end{cases}\). Il \textbf{problema di regolazione in feedback dall'errore} è il problema di determinare le matrici \(F,G,H\) del controllore tali che siano soddisfatte:
    \begin{itemize}
        \item \textbf{\emph{Stabilità} (S)}: Il sistema \(\begin{cases}
            \dot{x}=Ax+BH\chi \\
            \dot{\chi}=F\chi+GCx
        \end{cases}\) sia asintoticamente stabile;
        \item \textbf{\emph{Regolazione} (R)}: tutte le traiettorie del sistema \(\begin{cases}
            \dot{d}=Sd\\
            \dot{x}=Ax+BH\chi+Pd \\
            \dot{\chi}=F\chi+G(Cx+Qd)\\
            e=Cx+Qd
        \end{cases}\) siano tali che  \(\lim_{t\rightarrow\infty}e(t)=0\).
    \end{itemize}
\end{definition}
\section*{Problema di regolazione a Full Information}
Per poter risolvere il problema di regolazione a full information dobbiamo definire le seguenti ipotesi strutturali:\begin{itemize}
    \item Sia \(S\) la matrice dell'esosistema e \(\lambda\in\sigma{(S)}\), allora \(\forall\lambda\in\sigma{(S)},Re(\lambda)\geq 0\): ciò implica che \(\nexists d(0) \) tale che \(d(t)\) converge asintoticamente a zero. Se così non fosse \(d(t)\) non influisce sul comportamento asintotico del sistema e quindi basterebbe solamente stabilizzare il sistema per raggiungere l'obbiettico;
    \item Il sistema \(\dot{d}=Sd\) con \(d=0\) è raggiungibile: ciò implica che è possibile assegnare arbitrariamente gli autovalori di \(A+BK\)
\end{itemize}
In maniera preliminare dimostriamo l'equazione di Sylvester.
\begin{corollary}{\textbf{Equazione di Sylvester}}\\
    Sia \(A\in\mathbb{C}^{n\times n},B\in\mathbb{C}^{n\times n},C\in\mathbb{C}^{n\times n}\), l'equazione di Sylvester è una equazione matriciale lineare nella forma \(AX+XB=C\) con \(X\in\mathbb{C}^{n\times n}\). Valgono i seguenti enunciati:\begin{itemize}
        \item L'equazione di Sylvester ha soluzione se e solo se \(A\) e \(-B\) non hanno nessun autovalore in comune;
        \item L'equazione di Sylvester ha un'unica soluzione se \(A\) e \(-B\) non hanno autovalori in comune o un'unfinità di soluzioni composte da \(X=X_{0}+\hat{X}\) con \(X_{0}\) ottenuta da \(AX+XB=0\)
    \end{itemize}
    \begin{proof}
        L'equazione di Sylvester è equivalente al sistema lineare \(Gx=c\) in cui \(x= vec(X)=\begin{bmatrix}
            x_{11}\\x_{12}\\\vdots \\x_{21}\\x_{22}\\\vdots
        \end{bmatrix}\) e \(c=vec(X)=\begin{bmatrix}
            c_{11}\\c_{12}\\\vdots \\c_{21}\\c_{22}\\\vdots
        \end{bmatrix}\) e \(G=(I_{n}\bigotimes A)+(B^{T}\bigotimes I_{n})\). In particolare \(\bigotimes \) è detto \textbf{prodotto di Kronecker}:\(A\bigotimes B=\begin{bmatrix}
            a_{11}B&\cdots  &a_{1m}B\\
            \vdots & \vdots &\vdots \\
            a_{m1}B& \cdots & a_{mmB}
        \end{bmatrix}\). Questo sistema ha una unica soluzione se e solo se \(G\) non è singolare. Per una proprietà dell'operazione del \textbf{prodotto di Kronecker}, gli autovalori di \(G=(I_{n}\bigotimes A)+(B^{T}\bigotimes I_{n})\) sono \(\lambda_{A}+\lambda_{B},\forall \lambda_{A}\in\sigma{(A)},\lambda_{B}\in\sigma{(B)}\). Quindi, per non essere singolare non deve esistere nessun autovalore \(\lambda_{G}=0\). Quindi:
        \begin{equation*}
            \lambda_{A}+\lambda_{B}\neq 0\rightarrow \lambda_{A}\neq-\lambda_{B}\rightarrow \lambda_{A}\neq\lambda_{B}\forall \lambda_{A}\in\sigma{A},\lambda_{B}\in\sigma{B}
        \end{equation*}
    \end{proof}
\end{corollary}
\begin{theorem}
    Considerato il problema di regolazione a full information, supponiamo che \(\forall\lambda\in\sigma{(S)}:Re(\lambda)\geq 0\) e che \(\exists K,L\) tali che il sistema \(\dot{x}=(A+BK)x\) sia asintoticamente stabile, allora la condizione di regolazione è soddisfatta se e solo se \(\exists \Pi\in\mathbb{R}^{n\times r}\) tali che soddisfano le equazioni\begin{equation*}
        \begin{cases}
            \Pi S=(A+BK)\Pi+(P+BL)\\
            0=C\Pi+Q
        \end{cases}
    \end{equation*}
\end{theorem}
\begin{proof}
    Consideriamo il sistema \(\begin{cases}
        \dot{d}=Sd              \\
        \dot{x}=(A+BK)x+(P+BL)d \\
        e=Cx+QD
    \end{cases}\) e il cambio di coordinate \(\hat{d}=d,\hat{x}=x-\Pi d\)  con \(\Pi \) soluzione dell'equazione di Sylvester \( \begin{cases}
        \Pi S=(A+BK)\Pi+(P+BL)\\
        0=C\Pi+Q
    \end{cases}\).\newline
    Si nota che la soluzione è unica dato che:\begin{equation*}
        \begin{cases}
            \lambda\in\sigma{(A+BK)},Re(\lambda )< 0 \\
            \lambda\in\sigma{(S)},Re(\lambda )\geq 0
        \end{cases}\rightarrow\sigma{(A+BK)}\cap\sigma{(S)}=\{\varnothing \}\Rightarrow \forall(P+BL)\exists!\Pi
    \end{equation*}
    Riscrivendo il sistema nelle nuove coordinate:
    \begin{equation*}
        \begin{cases}
            \dot{\hat{x}}-\Pi \dot{\hat{d}}=(A+BK)\hat{x}+(A+BK)\Pi\hat{d}+(BL+P)\hat{d}\\
            \hat{e}=C\hat{x}+C\Pi\hat{d}+Q\hat{d}\\
            \dot{\hat{d}}=S\hat{d}
        \end{cases}
        \rightarrow^{Dal teorema}\begin{cases}
            \dot{\hat{x}}=(A+BK)\hat{x}\\
            e=C\hat{x}+(C\Pi+Q)\hat{d}\\
            \dot{\hat{d}}=S\hat{d}
        \end{cases}
    \end{equation*}
    Dalla stabilità sappiamo che \(\lim_{t\rightarrow\infty} \hat{x}(t)=0\) e dalla regolazione \(\lim_{t\rightarrow\infty}e(t)=0\leftrightarrow C\Pi+Q=0\). Ciò implica che anche per osillazioni di \(d,x\), si regolarizza la soluzione vincolandola sulla bisettrice del piano \(x,d\).
\end{proof}
Per fornire condizioni neessarie e sufficienti per la osluzione del problema di regolazione a full information occorre enunciare il seguente teorema.
\begin{theorem}{\textbf{Toerema FBI}}
    Considerato il problrma di regolazione a full information, supponiamo che la matrice \(S\) dell'esosistema abbia autovalori a parte reale positiva e il sistema \(\begin{cases}
        \dot{x}=Ax+Bu+Pd \\
        e=Cx+Qd
    \end{cases}\) con \(d=0\) sia raggiungibile. Allora esiste una legge di controllo a full information \(u=Kx+Ld\) che risolve il problema di regolazione se e solo se:\begin{equation*}
        \exists \Pi,\Gamma:\begin{cases}
            \Pi S=A\Pi+B\Gamma+P\\
            0=C\Pi+Q
        \end{cases}
    \end{equation*}
    \begin{proof}
        \(\Rightarrow{} \) Supponiamo di aver soddisfatto il problema di regolazione cioè che \(\exists \Pi,\Gamma \) tali che siano soddisfatte i requisiti di stabilità e di regolazione. Allora per il lemma si ha che:\begin{equation*}
            \exists\Pi\text{  tale che:}\begin{cases}
                (A+BK)\Pi+(P+BL)=A\Pi+B(K\Pi+L)+P\\
                0=C\Pi+Q
            \end{cases}
        \end{equation*}
        Quindi, noto \(\Pi \), si definisce \(\Gamma=K\Pi+L\).\newline
        \(\Leftarrow \)Supponiamo che \(\exists\Pi,\Gamma \) che risolvono \(\begin{cases}
            \Pi S=A\Pi+B\Gamma+P\\
            0=C\Pi+Q
        \end{cases}\), dobbiamo dimostrare che sono soddisfatte le ipotesi di stabilità e regolazione:\begin{itemize}
            \item \textbf{Stabilità:}\(K\) deve garantire stabilità asintotica per \(A+BK\) e in particolare \(K\) esiste sempre dato che abbiamo supposto che il sistema con \(d=0\) è raggiungibile;
            \item Sia \(L=\Gamma-K\Pi \), allora la coppia \((K,L)\) soddisfa il requisito di regolazione poiché:\begin{equation*}
                \begin{cases}
                    \Pi S=(A+BK)\Pi+(P+BL)=A\Pi+B\Gamma+P\\
                    0=C\Pi+Q
                \end{cases}
            \end{equation*}
        \end{itemize}
    \end{proof}
\end{theorem}
Il teorema appena enunciato implica che il controllo sia nella forma \(u=Kx+(\Gamma-K\Pi)d\) cioè composto da una parte di stabilizzazione e una di regolazione. Tuttavia, la matrice \(K\) esiste sempre, quindi la condizione di risolubilità del problema di regolazione a full information risiede nell'esistenza e nella risoluzione dell'equazione FBI in \(\Pi,\Gamma \). La condizione per cui il problema è risolubile è quindi fornita dal \textbf{lemma di Hautus}.
\begin{corollary}{\textbf{Lemma di Hautus}}
    L'equazioni FBI nelle incognite \(\Pi,\Gamma \) sono risolubili \(\forall P,Q
        \Leftrightarrow rank\left(\begin{bmatrix}
            sI-A & B \\C & 0
        \end{bmatrix}\right)=n+p \forall s\in\sigma(S)
    \)
\end{corollary}
Nel caso SISO, si ha che \(m=p=1\), si ha che \(rank\left(\begin{bmatrix}
    sI-A & B \\C & 0
\end{bmatrix}\right)=n+1 \forall s \text{zero per} \begin{cases}
    \dot{x}=Ax+Bu\\
    e=Cx
\end{cases}\)  Ovvero gli zeri di \( W(s)=C{(sI-A)}^{-1}B\)\newline
Ne deriva che il problrema della regolazione a full information per sistemi SISO è risolubile se e solo se gli autovalori del sistema esogne non sono zero di \(W(s) \) con ingresso \(u\) uscita \(e\) e segnale esogeno \(d=0\).
\section*{Problema di regolazione con retroazione dall'errore}
Per poter risolvere il problema di regolazione error feedback dobbiamo definire le seguenti ipotesi strutturali:\begin{itemize}
    \item Sia \(S\) la matrice dell'esosistema e \(\lambda\in\sigma{(S)}\), allora \(\forall\lambda\in\sigma{(S)},Re(\lambda)\geq 0\): ciò implica che \(\nexists d(0) \) tale che \(d(t)\) converge asintoticamente a zero. Se così non fosse \(d(t)\) non influisce sul comportamento asintotico del sistema e quindi basterebbe solamente stabilizzare il sistema per raggiungere l'obbiettico;
    \item Il sistema \(\dot{d}=Sd\) con \(d=0\) è raggiungibile: ciò implica che è possibile assegnare arbitrariamente gli autovalori di \(A+BK\)
    \item Il sistema \(\begin{bmatrix}
        \dot{x}\\\dot{d}
    \end{bmatrix}=\begin{bmatrix}
        A & P \\ 0 &S
    \end{bmatrix}\begin{bmatrix}
        x\\ d
    \end{bmatrix}, e=\begin{bmatrix}
        C & Q
    \end{bmatrix}\begin{bmatrix}
        x\\d
    \end{bmatrix}
    \) sia osservabile (se rilassato va bene anche la determinabilità). Ciò implica che la coppia \((A,C)\) è osservabile.
\end{itemize}
\begin{definition}{\textbf{Teorema}}\\
    Consideriamo il problema di regolazione in feedback dall'errore e supponiamo che le assuzioni siano soddisfatte. Allora, \(\exists \) un controllo in feedback dall'errore descritto da:\(\begin{cases}
        \dot{\chi }=F\chi+Ge\\
        u=H\chi
    \end{cases}\) che risolve il problema di relazione in error feedback se e solo se \(\exists \Pi,\Gamma \) tali che siano soddisfatte le equazioni:
    \begin{equation*}
        \begin{cases}
            \Pi S==A\Pi+B\Gamma+P\\
            0=C\Pi+Q
        \end{cases}
    \end{equation*}
\end{definition}
\begin{proof}
    \(\Rightarrow \) La necessità è uguale a quella del problema in full information;\newline
    \(\Leftarrow \) Supponaimo che \(\exists \Pi,\Gamma \) che soddistano il lemma di Hautus e quindi il problema in full information, Allora il controllore fornito sarà nella forma \(u=kx+Ld\) per cui \(\dot{x}=(A+BK)x\) sia asintoticamente stabile. Tuttavia, non è implementabile poiche si hanno solo misure di \(e(t)\). Quindi, cerchiamo di costruire asintoticamente delle stime \(\psi,\delta \) di \(x(t),d(t)\) ed implementiamo la legge di controllo \(u=K\psi+(\Gamma-K\Pi)\delta \).\newline In particolare, si considera un osservatore del tipo:\begin{equation*}
        \begin{bmatrix}
            \dot{\psi }\\
            \dot{\delta}
        \end{bmatrix}=\begin{bmatrix}
            A & O \\ 0 & S
        \end{bmatrix}\begin{bmatrix}
            \psi \\\delta
        \end{bmatrix}+\begin{bmatrix}
            G_{1}\\G_{2}
        \end{bmatrix}\left(\begin{bmatrix}
            C&Q
        \end{bmatrix}\begin{bmatrix}
            \psi \\\delta
        \end{bmatrix}-e\right)+\begin{bmatrix}
            B\\0
        \end{bmatrix}\begin{bmatrix}
            K & \Gamma-K\Pi
        \end{bmatrix}\begin{bmatrix}
            \psi \\\Gamma
        \end{bmatrix}
    \end{equation*}
    La stima degli errori \(e_{x}=x-\psi , e_{d}=d-\delta \) sono descritti dalla dinamica:\begin{equation*}
        \begin{bmatrix}
            \dot{e_{x}}\\\dot{e_{d}}
        \end{bmatrix}=\left(\begin{bmatrix}
            A & P\\ 0 &S
        \end{bmatrix}+\begin{bmatrix}
            G_{1}\\G_{2}
        \end{bmatrix}\begin{bmatrix}
            C & Q
        \end{bmatrix}\right)\begin{bmatrix}
            e_{x}\\e_{d}
        \end{bmatrix}
    \end{equation*}
    Poiché  per la terza assunzione \(\exists G_{1},G_{2}\) che assegnano gli autovalori del sistema di errore.\newline La legge di controllo può essere riscritta come \begin{equation*}
        u=Kx+(\Gamma-K\Pi)d-(Ke_{x}+(\Gamma-K\Pi)e_{d})
    \end{equation*}
    Il requisito di stabilità viene soddisfatto poiché il sistema  a ciclo chiuso è asintoticamente stabile dato che il primo e il secondo termine sono dovuti al full information vano a zero e il terzo per le impotesi fatti va anche a lui a zero. Di conseguenza si soddisfa anche la condizione di regolazione.
\end{proof}
\section*{Principio del modello interno}
Il controllore che risolve il problema di regolazione in feedback dall'errore è descritto da:\begin{equation*}
    \chi=\begin{bmatrix}
        \psi \\\delta
    \end{bmatrix}, F =\begin{bmatrix}
        A+G_{1}C+BK&P+G_{1}Q+B(\Gamma-K\Pi)\\
        G_{2}C&S+G_{2}Q
    \end{bmatrix}, H = \begin{bmatrix}
        K&\Gamma-K\Pi
    \end{bmatrix}
\end{equation*}
Questa legge di controllo soddista il requisito di stabilità e di regolazione.
\begin{definition}{\textbf{Principio del modello interno}}\\
    \(\exists\Sigma:rank(\Sigma )=r \) è tale che \(F\Sigma = \Sigma S\). In particoalre, qualsiasi autovalore di S è anche autovalore di F.
\end{definition}
\begin{proof}
    Sia \(\begin{bmatrix}
        \Pi \\ I
    \end{bmatrix}\) e sia \(rank(\Sigma )=r\), per costruzione si ha che:\begin{equation*}
        F\Sigma=\begin{bmatrix}
            A+G_{1}C+BK&P+G_{1}Q+B(\Gamma-K\Pi)\\
            G_{2}C&S+G_{2}Q
        \end{bmatrix}\begin{bmatrix}
            \Pi \\ I
        \end{bmatrix}=\begin{bmatrix}
            \Pi S\\S
        \end{bmatrix}=\Sigma S
    \end{equation*}
    Ore, sia \(\lambda \) autovalore di S e \(v\) il suo corrispondente autovettore. Allora valore che:\begin{equation*}
        Sv = \lambda v\rightarrow F\Sigma v = \Sigma S v = \lambda\Sigma v
    \end{equation*}
    Cioè è un autovalore di F.
\end{proof}
Il principio del modello interno può essere visto nel seguente modo:\textbf{la legge di controllo deve possedere una copia dell'esosistema}.
\end{document}