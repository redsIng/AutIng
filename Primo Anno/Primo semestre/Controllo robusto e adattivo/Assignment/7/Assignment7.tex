\documentclass{beamer}
\mode<presentation> {
    \usetheme{Copenhagen} \usecolortheme{whale}
    }
\usepackage{graphicx}
\usepackage{multicol}
\usepackage{subfig}
\usepackage{hyperref}

\usepackage{cleveref}

\setbeamersize{text margin left=3mm,text margin right=5mm }
\geometry{paperwidth=650pt, paperheight=400pt}
\usepackage{verbatim}
\graphicspath{{Images/}}
\title[Assignments 7]{Pendolo Inverso}
\author{Lorenzo Rossi Matricola: 0301285}
\begin{document}
\begin{frame}
	\titlepage{}
\end{frame}
\begin{frame}
	\begin{columns}[t]
		\begin{column}{.5\textwidth}
			\tableofcontents[sections={1-5}] % chktex 8
		\end{column}
		\hspace{-1cm}
		\begin{column}{.5\textwidth}
			\tableofcontents[sections={6-11}] % chktex 8
		\end{column}
	\end{columns}
\end{frame}
\begin{frame}
	\frametitle{Formulazione del Sistema}
	\section{A0 - Formulazione del sistema}% chktex 8
	Noti: \(M=1 kg,L=1m,F=1\frac{Kg}{s},g=9.81\frac{m}{s^2}\)\newline
	\begin{tabular}{c|c} % chktex 44
		\begin{minipage}{0.55\textwidth}
			\begin{equation*}
				\begin{cases}
					M\ddot{s}+F\dot{s}-\mu=d_{1} \\
					\ddot{\phi}-\frac{g}{L}\sin{(\phi)}+\frac{1}{L}\ddot{s}\cos{(\phi)}=0
				\end{cases}
			\end{equation*}
			Esplicitando \(\ddot{s},\ddot{\phi}\) si ottiene;\small
			\begin{equation*}
				\begin{cases}
					\ddot{s}=-\frac{F}{M}\dot{s}+\frac{1}{M}\mu+\frac{1}{M}d_1 \\
					\ddot{\phi}=\frac{g}{L}\sin{(\phi)}-\frac{1}{L}\ddot{s}\cos{(\phi)}=\frac{g}{L}\sin{(\phi)}+\frac{1}{L}(\frac{F}{M}\dot{s}-\frac{1}{M}\mu-\frac{1}{M}d_1)
				\end{cases}
			\end{equation*}
			Sia:
			\begin{equation*}
				x=\begin{bmatrix}
					x_{1} & x_{2} & x_{3} & x_{4}
				\end{bmatrix}^{T}=\begin{bmatrix}s & \dot{s} & \phi & \dot{\phi}
				\end{bmatrix}^{T}
			\end{equation*}
			\begin{equation*}
				u=\begin{bmatrix}
					u_{1} & u_{2}
				\end{bmatrix}^{T}=\begin{bmatrix}\mu & d_{1}
				\end{bmatrix}^{T}
			\end{equation*}
		\end{minipage} &
		\begin{minipage}{0.40\textwidth}
			Si ottiene il sistema finale:\small
			\begin{equation*}
				\begin{cases}
					\dot{x_{1}}=x_2                                                 \\
					\dot{x_{2}}=-\frac{F}{M}x_{2}+\frac{1}{M}u_{1}+\frac{1}{M}u_{2} \\
					\dot{x_{3}}=x_{4}                                               \\
					\dot{x_{4}}=\frac{g}{L}\sin{(x{3})}+\frac{1}{L}(\frac{F}{M}x_{2}-\frac{1}{M}u_{1}-\frac{1}{M}u_{2})\cos{(x{3})}
				\end{cases}
			\end{equation*}
		\end{minipage}
	\end{tabular}
\end{frame}
\begin{frame}
	\frametitle{A1 - Punti Equilibrio}% chktex 8
	\section{A1 - Punti di Equilibrio}% chktex 8
	\textbf{Calcolare tutte i punti di equilibrio del sistema per \(\mu = d_{1}{(t)} = 0\)}
	Si impone \(\dot{x}=\textbf{0}\):
	\begin{equation*}
		\begin{cases}
			\dot{x_{1}}=0 \\
			\dot{x_{2}}=0 \\
			\dot{x_{3}}=0 \\
			\dot{x_{4}}=0
		\end{cases}
		\rightarrow
		\begin{cases}
			0=x_{2} \\
			0=0     \\
			0=x_{4} \\
			0=\sin{x_{3}}
		\end{cases}
		\rightarrow
		\begin{cases}
			s:x_{1}\in\mathbb{R}        \\
			\dot{s}:x_{2}=0             \\
			\phi:x_{3}=0 \lor x_{3}=\pi \\% chktex 21
			\dot{\phi}:x_{4}=0
		\end{cases}
	\end{equation*}
	In particolare, si ha un punto di equilibrio nei casi in cui:\begin{itemize}
		\item Le velocità del carrello e del pendolo sono nulle;
		\item Il pendolo è perpendicolare al piano.
		\item Qualisiasi posizione del piano su cui si muove il carrello è un punto di equilibrio.
	\end{itemize}
\end{frame}
\begin{frame}
	\frametitle{A2 - Linearizzazione del sistema} % chktex 8
	\section{A2 - Linearizzazione del sistema} % chktex 8
	\textbf{Scrivere le equazioni del sistema linearizzato attorno al punto di equilibrio \(\phi=s=\dot{\phi}=\dot{s}=0\)}
	Per ottenere la linearizzazione occorre imporre che:\begin{equation*}
		A_{lin}=\nabla_{x}f(x,u)\lvert_{x=0,u=0}\quad
		B_{lin}=\nabla_{u}f(x,u)\lvert_{x=0,u=0}\quad
	\end{equation*}
	Quindi, effettuando le derivate si giunge a:
	\begin{equation*}
		A_{lin}=\begin{bmatrix}
			0 & 1            & 0           & 0 \\
			0 & -\frac{F}{M} & 0           & 0 \\
			0 & 0            & 0           & 1 \\
			0 & \frac{F}{LM} & \frac{g}{L} & 0
		\end{bmatrix}
		\quad B_{lin}=\begin{bmatrix}
			0 & 0 \\\frac{1}{M}&\frac{1}{M}\\0&0\\-\frac{1}{LM}&-\frac{1}{LM}
		\end{bmatrix}
	\end{equation*}
	\begin{equation*}
		\dot{\tilde{x}}=A_{lin}\tilde{x}+B_{lin}u
	\end{equation*}
\end{frame}
\begin{frame}
	\frametitle{A3 - Forma standard nello spazio di stato}% chktex 8
	\section{A3 - Forma standard nello spazio di stato}% chktex 8
	\textbf{Scrivere il sistema lineare nella forma:\begin{equation*}
			\dot{x}=Ax+Bu+PD,\quad y=Cx
		\end{equation*}}
	\label{sec:A3}% chktex 24
	Siano:\begin{equation*}
		x(t)=\begin{bmatrix}
			s(t) & \dot{s(t)} & \phi(t) & \dot{\phi(t)}
		\end{bmatrix}^{T}\quad u(t)=\mu(t)\quad y(t)=\begin{bmatrix}
			s(t) & \phi(t)
		\end{bmatrix}^{T}
	\end{equation*}
	Quindi:\small
	\begin{equation*}
		\dot{x}=\begin{bmatrix}
			0 & 1            & 0           & 0 \\
			0 & -\frac{F}{M} & 0           & 0 \\
			0 & 0            & 0           & 1 \\
			0 & \frac{F}{LM} & \frac{g}{L} & 0
		\end{bmatrix}x+\begin{bmatrix}
			0 \\\frac{1}{M}\\0\\-\frac{1}{LM}
		\end{bmatrix}u+\begin{bmatrix}
			0 \\\frac{1}{LM}\\0\\-\frac{1}{LM}
		\end{bmatrix}d_{1}\quad y=\begin{bmatrix}
			1 & 0 & 0 & 0 \\0&0&1&0
		\end{bmatrix}x
	\end{equation*}
\end{frame}
\begin{frame}
	\frametitle{A4 - Controllabilità}% chktex 8
	\section{A4 - Controllabilità}% chktex 8
	\textbf{Mostra che la coppia \((A,B)\) è controllabile}
	A tempo continuo vale che:
	\begin{block}{}
		\((A,B) \) controllabile \(\Longleftrightarrow \) \((A,B) \) raggiungibile  \(\Longleftrightarrow
		rank{(R)}=n \quad n=\dim{(A)},\quad R= \begin{bmatrix}
			B\ AB\ \dots A^{n-1}B
		\end{bmatrix}\)
	\end{block}
	\begin{equation*}
		rank(R)=rank\left(\begin{bmatrix}
			0             & \frac{1}{M}      & -\frac{F}{M^2}                         & \frac{F^{2}}{M^{3}}                        \\
			\frac{1}{M}   & -\frac{F}{M^2}   & \frac{F^{2}}{M^{3}}                    & -\frac{F^{3}}{M^{4}}                       \\
			0             & -\frac{1}{LM}    & \frac{F}{LM^{2}}                       & -\frac{g}{L^{2}M}-\frac{F^{2}}{LM^{3}}     \\
			-\frac{1}{LM} & \frac{F}{LM^{2}} & -\frac{g}{L^{2}M}-\frac{F^{2}}{LM^{3}} & \frac{Fg}{L^{2}M^{2}}+\frac{F^{3}}{LM^{4}} \\
		\end{bmatrix}\right)=4
	\end{equation*}
	La coppia \((A,B) \) è controllabile.
\end{frame}
\begin{frame}
	\frametitle{A5 - Problema di Regolazione}% chktex 8
	\section{A5 - Problema di Regolazione}% chktex 8
	Per formulare un problema di regolazione si procede nel seguente modo:
	\begin{equation*}
		\dot{d_{1}}=0\rightarrow \dot{d_{1}}=S_{1}d_{1}\text{ con} S_{1}=\begin{bmatrix}
			0
		\end{bmatrix}\Longleftrightarrow d_{1}(t)=d_{1}(0),\forall t\geq 0
	\end{equation*}
	\begin{equation*}
		d_{2}=\alpha\sin{(\omega t)}\rightarrow \dot{\overline{d_2}}=S_{2}\overline{d_{2}}\quad S_{2}=\begin{bmatrix}
			0       & \omega \\
			-\omega & 0
		\end{bmatrix},\overline{d_{2}}(0)=\begin{bmatrix}
			0 \\\alpha
		\end{bmatrix}\Rightarrow d_{2}(t)=\begin{bmatrix}
			1 & 0
		\end{bmatrix}\overline{d_{2}}(t)
	\end{equation*}
	Con l'introduzione del segnale \(d_{3}(t)\) è possibile esprimere i segnali esogeni tramite:
	\begin{equation*}
		\dot{d}=Sd=\begin{bmatrix}
			0 & 0 & 0 \\ 0&0&\omega \\0&-\omega & 0
		\end{bmatrix}d\quad d(t)=\begin{bmatrix}
			d_{1}(t) \\d_{2}(t)\\d_{3}(t)
		\end{bmatrix},d(0)=\begin{bmatrix}
			const \\0\\\alpha
		\end{bmatrix}
	\end{equation*}
	Infine, definendo \(e(t)=x{1}(t)-d_(t)=s(t)-d_{2}(t)=s(t)-\alpha\sin{(\omega t)}\), si riscrive il sistema A3.

\end{frame}
\begin{frame}
	\frametitle{A5.1 - Problema di Regolazione}% chktex 8
	Il nuovo sistema viene descritto dalle seguenti equazioni:
	\begin{equation*}
		\begin{cases}
			\dot{x}=Ax+Bu+Pd \\
			y=Cx             \\
			e=C_{e}x+Qd      \\
			\dot{d}=Sd
		\end{cases}
	\end{equation*}
	In cui:\begin{equation*}
		A=\begin{bmatrix}
			0 & 1            & 0           & 0 \\
			0 & -\frac{F}{M} & 0           & 0 \\
			0 & 0            & 0           & 1 \\
			0 & \frac{F}{LM} & \frac{g}{L} & 0
		\end{bmatrix},\quad B=\begin{bmatrix}
			0 \\\frac{1}{M}\\0\\-\frac{1}{LM}
		\end{bmatrix}\quad P=\begin{bmatrix}
			0 & 0 & 0 \\\frac{1}{LM}&0&0\\0&0&0\\-\frac{1}{LM}&0&0
		\end{bmatrix}
		C=\begin{bmatrix}
			1 & 0 & 0 & 0 \\0&0&1&0
		\end{bmatrix},\quad C_{e}=\begin{bmatrix}
			1 & 0 & 0 & 0
		\end{bmatrix}\quad S=\begin{bmatrix}
			0 & 0 & 0 \\ 0&0&\omega \\0&-\omega & 0
		\end{bmatrix}
	\end{equation*}
	\begin{center}
		\begin{equation*}
			Q=\begin{bmatrix}
				0 & -1 & 0
			\end{bmatrix}
		\end{equation*}
	\end{center}
\end{frame}
\begin{frame}
	\frametitle{A6 - Legge di controllo a full information}% chktex 8
	\section{A6 - Legge di controllo a full information}% chktex 8
	\textbf{Consideriamo il problema di regolazione A5. Mostra che il problema è risulubile tramite una legge di controllo a full information}
	In un problema di regolazione a full information vogliamo determinare una legge di controllo \(u=Kx+Ld,L=\Gamma -K\Pi \) tale che:\begin{itemize}
		\item \textbf{S}: Il sistema \(\dot{x}=(A+BK)x\) sia asintoticamente stabile;
		\item \textbf{R}: Tutte le traiettorie del sistema:\begin{equation*}
			      \begin{cases}
				      \dot{x}=(A+BK)x+(BL+P)d \\
				      y=Cx                    \\
				      e=C_{e}x+Qd             \\
				      \dot{d}=Sd
			      \end{cases}
		      \end{equation*}
		      sono tali che \(\lim_{t\rightarrow \infty}e(t)=0\).
	\end{itemize}
	Per il teorema del problema di regolazione a full information FBI, esiste una legge di controllo a full information se e solo se \(\exists \Pi ,\Gamma \) tale che siano soddisfatte le equazioni:\(\Pi S= A\Pi +B\Pi+P\quad 0=C\Pi+Q\).
	Inoltre, dal lemma di Hautus si ha che il teorema FBI è soddisfatto \(\forall P,Q\) se e solo se:\begin{equation*}
		rank\left(\begin{bmatrix}
			sI-A & B \\C&0
		\end{bmatrix}\right)=n+p\quad \forall s\in \sigma(S)
	\end{equation*}
\end{frame}
\begin{frame}
	\frametitle{A6.1 - Legge di controllo a full information}% chktex 8
	Quindi:\begin{equation*}
		rank\left(\begin{bmatrix}
				s & -1            & 0            & 0  & 0             \\
				0 & s+\frac{F}{M} & 0            & 0  & \frac{1}{M}   \\
				0 & 0             & s            & -1 & 0             \\
				0 & -\frac{F}{LM} & -\frac{g}{L} & s  & -\frac{1}{LM} \\
				1 & 0             & 0            & 0  & 0
			\end{bmatrix}\right)=5,\forall{s}\in\sigma(S)\Longrightarrow \text{Equazioni FBI rispettate}
	\end{equation*}
\end{frame}
\begin{frame}
	\frametitle{A7 - Legge di controllo in feedback dall'errore}% chktex 8
	\section{A7 - Legge di controllo in feedback dall'errore}% chktex 8
	Sia \(e_{0}=\begin{bmatrix}
		e \\\phi
	\end{bmatrix}=\begin{bmatrix}
		s-d_{2} \\\phi
	\end{bmatrix}=Cx+Q_{0}d\) con \(Q_{0}=\begin{bmatrix}
		- & Q & - \\0&0&0
	\end{bmatrix}=\begin{bmatrix}
		0 & -1 & 0 \\0&0&0
	\end{bmatrix}\). Questo segnale è necessario alla realizzazione dell'osservatore che produce le stime \(\zeta(t),\delta(t)\), rispettivamente di \(x(t)\) e\( d(t)\), necessarie alla generazione del controllo \(u(t)\). Quindi il controllore dinamico risultante è del tipo:
	\begin{equation*}
		\begin{cases}
			\dot{\chi}=F\chi+Ge_{0} \\
			u=H\chi
		\end{cases}\quad
		\chi=\begin{bmatrix}
			\zeta \\ \delta
		\end{bmatrix}
	\end{equation*}
	Inoltre, deve essere tale che:
	\begin{itemize}
		\item \textbf{S}:\begin{equation*}
			      \begin{cases}
				      \dot{x}=Ax+BH\chi \\
				      \dot{\chi}=F\chi+GCx
			      \end{cases}
		      \end{equation*}\\ asintoticamente stabile;
	\item \textbf{R}:
	\begin{tabular}{c c}
		\(\begin{aligned}
			\begin{cases}
				\dot{x}=Ax+BH\chi+Pd          \\
				\dot{\chi}=F\chi+G(Cx+Q_{0}d) \\
				y=Cx                          \\
				e=C_{e}x+Qd                   \\
				\dot{d}=Sd
			\end{cases}
		\end{aligned}\) &
		\(\begin{aligned}
			F=\begin{bmatrix}
				A+G_{1}C+BK & P+G_{1}Q_{0}+BL \\
				G_{2}C      & S+G_{2}Q_{0}
			\end{bmatrix}\quad H=\begin{bmatrix}
				K & L
			\end{bmatrix}\quad G=-\begin{bmatrix}
				G_{1} \\G_{2}
			\end{bmatrix}\quad L=\Gamma-K\Pi
		\end{aligned}\)
		\end{tabular}\\
	con le traiettorie \(\lim_{t\rightarrow \infty }e(t)=0 \)
\end{itemize}
\end{frame}
\begin{frame}
	\frametitle{A7.1 - Legge di controllo in feedback dall'errore }% chktex 8
	Per la realizzazione dell'osservatore occorre verificare che la coppia \(A_{0},C_{0}=\left(\begin{bmatrix}
		A&P\\0&S
	\end{bmatrix},\begin{bmatrix}
		C&Q_{0}
	\end{bmatrix}\right)\). In particolare:
	\begin{block}{}
		\begin{center}
			\((A,C)\) osservabile \(\Longleftrightarrow rank(O)=n\) con \(n=\dim{(A_{0})}=7,O=\begin{bmatrix}
				C_{0}\\C_{0}A_{0}\\\vdots \\C_{0}A_{0}^{n-1}
			\end{bmatrix}\)
		\end{center}
	\end{block}
	Nel nostro caso \(rank(O)=7\Rightarrow (A_{0},C_{0})\) è osservabile. Inoltre, bisogna soddisfare i requisiti di regolazione. Per cui, definito \(e=C_{e}x+Qd=s-d_{2}\) si ha che:
	\begin{itemize}
		\item Dal teorema FBI:\(\exists{F,G,H}\)tali che si rispettino le condizioni \textbf{S},\textbf{R} se e solo se:\begin{equation*}
			\exists{\Gamma,\Pi}:\begin{cases}
				\Pi S= A\Pi +B\Pi+P\\
				 0=C\Pi+Q
			\end{cases}
		\end{equation*}
		\item Dal lemma di Hautus:\begin{equation*}
			\exists{\Gamma,\Pi}:\begin{cases}
				\Pi S= A\Pi +B\Pi+P\\
				 0=C\Pi+Q
			\end{cases}\forall P,Q\Longleftrightarrow rank\left(\begin{bmatrix}
				sI-A&B\\C_{e}&0
			\end{bmatrix}\right)=n+p=5\forall s\in\sigma(S)
		\end{equation*}
	\end{itemize}
\end{frame}
\begin{frame}
	\frametitle{B1 - Legge di controlo full information A5}% chktex 8
	\section{B1 - Legge di controlo full information A5}% chktex 8
	La legge di controlo full information si ottiene dalle equazioni FBI.In particolare, si ottiene che:% chktex 13
	\begin{equation*}
		\Pi=\begin{bmatrix}
			0&1&0\\0&0&\omega \\0&-\frac{\omega^{2}}{L\omega^{2}+g}&0\\0&0&-\frac{\omega^{3}}{L\omega^{2}+g}
		\end{bmatrix}\quad \Gamma=\begin{bmatrix}
			-1&-M\omega^{2}&F\omega
		\end{bmatrix}
	\end{equation*}
	Inoltre, la legge di controllo \(u=Kx+(\Gamma-K\Pi )d\) viene scelta con \(K\) tale che la matrice \(A+BK\) abbia gli autovalori desiderati per cui si ottiene stabilità asintotica. L'assegnazione degli autovalori può essere effettuata tramite la formula di Ackermann o di Mitter.
\end{frame}
\begin{frame}
	\frametitle{B2+B3 - Simulazioni \(\omega=0.1\): Lineare e Non Lineare}% chktex 8
	\section{B2+B3 - Simulazioni}% chktex 8
	

\end{frame}
\begin{frame}
	\frametitle{B4 - Simulazione \(\omega\in \{1,10\} \): Lineare e Non Lineare}% chktex 8
	\section{B4 - Simulazioni}% chktex 8
	

\end{frame}
\end{document}